\documentclass{article}
\title{On Inclusive Fandom: Addressing Harassment and Boundary Violations}
\author{Nova (nova@novalinium.com)}
\date{Created November 9th, 2018; Last Revised November 8th, 2018 (Revision 0.1)}
\begin{document}
\maketitle
\section{Scope}
This document consists of a description of concerns and possible solutions to those concerns at Balticon, the Baltimore Science Fiction Society (BSFS), and in fandom in general. In this document, concerns are limited to problems affecting the ability of fans to participate in community, specifically around harassment behaviors. It is not intended to address issues that may prevent people from identifying as fans in the first place, nor is it intended to address accessibility (a11y) or internationalization (i18n). It is intentionally simplistic in its approach and is not intended as a comprehensive perscriptive policy, but as recommendations based on experience with other groups.

\subsection{Principles}
The guiding principles governing which concerns are identified and by which means they may be addressed are to:
\begin{enumerate}
    \item Improve truth-finding in response to incidents
    \item Lower incidence of traumatic events at or about community events (cons, meetups etc)
    \item Increase trust in organizational competence for handling incidents
    \item Ensure that adequate response is taken, and mitigate or limit recurring incidents
    \item Intentionality in gendered responses
\end{enumerate}

\subsection{Constraints}
Some constraints on recommendations made by this document include:
\begin{enumerate}
    \item Available personnel-hours
    \item Propriety in investigations
    \item Conflicts of interest
    \item Fallibility of human judgment and the author
\end{enumerate}

This document will not address these concerns in the context of feminist theory, as I feel that this would make it more difficult for recommendations to be more broadly accepted.

\section{Concerns}
\subsection{Background}
There are dueling narratives in fan spaces, just as there are in larger society with regards to harassment, and often differences in perception come down to cultural differences. These differences often distract from issues at hand, and make tractable problems into intractable ones (see "Politics is the Mind-Killer", Yudkowsky 2007). This effort sink creates discourse mires that turn away people who might be otherwise interested in bridging gaps. Why might this be so, and what can be done about it?

I assert that this is true in science-fiction fandom most of all: this fandom is so often a futuristic pursuit, and yet in its activities and manners it confines itself to the imperfect methods of the past due to unwillingness or perceived inability to address its place in a broader societal context. Indeed, in a community that strives to glean glimpses of the future, no stranger to the self-deception that a narrative can railroad one into, it is surprising how quickly and easily this self-deceptive tendency turns Ouroborous and leads fans to believe that in fact, there are no problems at all.

I believe that this is a narrative that it is not only possible, but necessary to break free from, and it is imperative to set aside the dominant narratives we each contextualize our own experiences in to create inclusive solutions utilizing community ingenuity: the true spirit of fandom. This can be done by intentionality in discussion, acknowledgement of the biases we each bring to the table, and a shared understanding that end of the day, we are all here first and foremost as fans. I urge adoption of the virtue of charity in discussion: if one does not assume the best from the parties in discussion, then discussion turns into debate, and debate turns into factiousness, a sorry fate indeed.

Most concerns can be divided into two categories: culture concerns and procedure concerns. This division corresponds roughly to "why do we do this" and "how do we do this" questions.

\subsection{Culture}
\subsection{Procedure}
A common complaint be all parties involved in harassment complaints is a lack of transparency in the process. There is often a lack of understanding as to the flow through which issues are resolved, which leads to confusion and frustration.
\end{document}
